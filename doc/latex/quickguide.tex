{% -*- mode: LaTeX; TeX-PDF-mode: t; TeX-master: "manual"; -*-
}

\chapter{Quick Guide to \ei}
\label{ch:quickguide}

This chapter provides a quick introduction on \emph{how to integrate
  an application in the \ei framework}. In particular, we develop a
simple application, integrate it in the \ei server, and try it out
through the web-client.
%
The presentation is incremental, we start with a simple application
and in each step we add more features to demonstrate the different
parts of the \ei framework.
%
In our explanation we assume that a Unix based operating system is
used, however, we comment on how to do the analog operations on
Windows when they are different.
%
Note that in this chapter we only use the web-client, for other
clients refer to Chapter~\ref{ch:clients}.
%
We assume that \ei is already installed and working, which can be done
following the instructions available in
\texttt{\href{http://github.com/abstools/easyinterface}{INSTALL.md}}.
%

Let us start by trying some demo applications that are available by
default in the web-client.
%
If you visit \url{http://localhost/ei/clients/web}, you should get a
page similar to the one shown in Figure~\ref{fig:webclient} on
Page~\pageref{fig:webclient}.
%
At the top part of this page you can see a button with the label
\applybutton, and to its right a combo-box with several items
\texttt{Test-0}, \texttt{Test-1}, etc. These items correspond to
applications available in the web-client, and we will refer to it as
the applications menu.
%
To the left of \applybutton there is a button with the label
\settingbutton, if you click it you will see that each \texttt{Test-i}
has also some parameters that can be set to some values.
%
Note that, by default, the web-client is configured to connect to the
\ei server at \url{http://localhost/ei/server} and ask for all
applications, together withe their corresponding parameters, that are
available at that server.
%
Note also that application \texttt{Test-i} actually corresponds to the
bash-script \texttt{server/bin/default/test-i.sh}, and that its
configuration file is \texttt{server/config/default/test-i.cfg} (later
you will understand the details of such configuration files).
%

If you select an application, from the combo-box, and click on
\applybutton, the web-client sends a request to the server to execute
this application. The request includes also the current values of the
parameters (those in the settings section) and the file that is
currently active in the code editor area.
%
The server, in turn, executes the corresponding program, i.e., the
bash-script \texttt{server/bin/default/test-i.sh} in this case, and
redirects its output back to the web-client.  
%
The web-client will either print this output in the console area, or
view it graphically if it uses the \ei output language. Execute the
demo applications just to get an idea on which graphical output we are
talking about (e.g., highlight text, markers).
%
In the rest of this chapter we explain, step by step, how to add your
own application to \ei.


\section{Add Your First Application to the \ei Server}

When we add an application to the \ei server it will appear
automatically in the applications menu of the web-client (unless you
have changed the configuration of the web-client already!).
%
Let us add a simple ``Hello World'' application.

We start by creating a bash-script that represents the executable of
our application (it could be any other executable). We will place this
bash-script in the directory \texttt{server/bin/default} together with
the \texttt{test-i.sh} scripts, however, this is not obligatory and it
can be placed anywhere in the file system as far as the HTTP server
can access it.
%
Create a file \texttt{myapp.sh} in \texttt{server/bin/default} with
the following content:

\medskip
\begin{lstlisting}[style=script]
#!/bin/bash

echo "Hello World!"
\end{lstlisting}

\medskip
\noindent
As you can see, it is a simple program that prints "Hello World" on
the standard output. Later we will see how to pass input to this
application and how to generate more sophisticated output.
%
Change the permissions of \texttt{myapp.sh} by executing the following
(on Windows this is typically not needed):

\medskip
\begin{lstlisting}
> chmod -R 755 myapp.sh
\end{lstlisting}

\medskip
\noindent
Execute \texttt{myapp.sh} (in a shell) to make sure that it works
correctly before proceeding to the next step.

Next we will configure the server to recognize our application.
%
Create a file \texttt{myapp.cfg} in the directory
\texttt{server/config/default} with the following content (we could
place this file anywhere under \texttt{server/config} not necessarily
in \texttt{default}):

\medskip
\begin{lstlisting}
<app visible="true">
  <appinfo>
    <acronym>MFA</acronym>
    <title>My First Application</title>
    <desc>
      <short>A simple EI application</short>
      <long>A simple application that I've done using the EasyInterface Framework</long>
    </desc>
  </appinfo>
  <apphelp>
    <content format='html'>
      This is my first <b>EasyInterface</b> application!
    </content>
  </apphelp>
  <execinfo>
    <cmdlineapp>(*./default/myapp.sh*)</cmdlineapp> % (* *) are used to escape
  </execinfo>
</app>
\end{lstlisting}

\medskip
\noindent
Let us explain the meaning of the different elements of this
configuration file.
%
The \lst{app} tag is used to declare an \ei application, and its
\lst{visible} attribute tells the server to list this application when
someone asks for the list of installed applications. Changing this
value to \lst{"false"} will make the application ``hidden'' so only
those who know its identifier can use it.
%
The \lst{appinfo} tag provides general information about the
application, this will be used by the clients to show the application
name, etc.
%
The \lst{apphelp} tag provides some usage information about the
application, or simply provide a link to another page where such
information can be found. The actual content goes inside the
\lst{content} tag, which is HTML as indicated by the \lst{format}
attribute (use 'text' for plain text).
%
The most important part is the \lst{execinfo} tag, which provides
information on how to execute the application. 
%
The text inside \lst{cmdlineapp} is interpreted as a command-line
\emph{template}, such that when the server is requested to execute the
corresponding application it will simply execute this command-line and
redirect its output back to the client. Later you will understand why
we call it \emph{template}. 
%
Note that before executing the script, the server changes the current
directory to \texttt{server/bin} and thus the command-line should be
relative to \texttt{server/bin}.

Next we add the above configuration file to the server. This is done
by adding the following line to \texttt{server/config/default/apps.cfg}
(inside the \lst{apps} tag):

\medskip
\begin{lstlisting}
 <app id="myapp" src="default/myapp.cfg" />
\end{lstlisting}

\medskip
\noindent
Here we tell the server that we want to install an application as
defined in \texttt{myapp.cfg}, and we want to assign it the
\emph{unique} identifier \lst{"myapp"}. This identifier will be mainly
used by the server and the clients when they communicate, we are not
going to use it anywhere else. 
%
Note that in \texttt{default/apps.cfg} we used
\lst{"default/myapp.cfg"} and not \lst{"myapp.cfg"}. This is because
the server looks for configuration files starting in
\texttt{server/config}.
%
Note also that the main configuration file of the \ei server is
\texttt{server/config/eiserver.default.cfg}, and that
\texttt{default/apps.cfg} is imported into that file (open
\texttt{server/config/eiserver.default.cfg} to see this).

Let us test our application. Go back to the web-client and reload the
page, you should see a new application named \texttt{MFA} in the
applications menu. If you click on the \helpbutton button you will see
the text provided inside the \lst{apphelp} tag above.
%
Select this application and click on the \applybutton button, the
message "Hello World!" will printed in the console area.


\section{Passing Input Files to Your Application}

Applications typically receive input files (e.g., programs) to
process. You must have noticed that the web-client provides the
possibility of creating and editing such files.  In this section we
explain how to pass these files, via the server, to our application
when the \applybutton button is clicked.

When you click on the \applybutton button the web-client passes the
currently opened file (i.e., the content of the active tab) to the
server, and if you use the \applybutton option from the context menu
of the file-manager (select an element from the files tree on the left,
and use the mouse right-click to open the context menu) it passes all
files in the corresponding sub-tree.
%
What is left is to tell the server how to pass these files to our
application. Let us assume that \texttt{myapp.sh} is prepared to
receive input files as follows:

\medskip
\begin{lstlisting}
> myapp.sh -f file1.c file2.c file3.c
\end{lstlisting}

\medskip 
\noindent
In order to tell the server to pass the input files (that were
received from the client) to \texttt{myapp.sh}, open
\texttt{myapp.cfg} and change the command-line template to the
following:

\medskip
\begin{lstlisting}
<cmdlineapp>(*./bin/default/myapp.sh*) -f _ei_files</cmdlineapp>
\end{lstlisting}

\medskip
\noindent
When the server receives the files from the client, it stores them in
a temporary directory, e.g., in \texttt{/tmp}, replaces \lst{_ei_files}
by the list of their names, and then execute the resulting
command-line (now you see why we call it command-line template).
%
It is important to note that only \lst{_ei_files} changes in the above
template, the rest remain the same. Thus, the parameter
``\texttt{-f}'' means nothing to the server, we could replace it by
anything else or even completely remove it --- that depends only on
how our application is programmed to receive input files.

Let us now change \texttt{myapp.sh} to process the received files in
some way, .e.g., to print the number of lines in each file. For this,
replace the content of \texttt{myapp.sh} by the following:

\medskip
\begin{lstlisting}[style=script]
#!/bin/bash

. default/parse_params.sh

echo "I've received the following command-line parameters:"
echo ""
echo "  $@"

echo ""
echo "File statistics:"
echo ""
for f in $files 
do
   echo " - $f has " `wc -l $f | awk '{print $1}'` "lines"
done
\end{lstlisting}
%$

\medskip
\noindent
Let us explain the above code. 
%
At line 3 we executes an external bash-script to parse the
command-line parameters, the details are not important and all you
should know is that it leaves the list of files (that appear after
\lst{-f}) in the variable \lst{files}.
%
Lines 5-7 print the command-line parameters, just to give you an idea
how the server called \texttt{myapp.sh}, and the loop at lines 12-15
traverses the list of files and prints the number of lines in each
one.

Let us test our application. First run \texttt{myapp.sh} from a shell
passing it some existing text files, just to check that it works
correctly.
%
Then go back to the web-client, reload the page, select \texttt{MFA}
from the applications menu, open a file from the file-manager, and
finally click the \applybutton button. Alternatively, you can also
select an entry from the file-manager and choose \applybutton from its
context menu. You should see the output of the application in the
console area.

\section{Passing Outline Entities to Your Application}
\label{ch:quickguide:outline}

In the web-client, the area on the right is called the outline area
(see Figure~\ref{fig:webclient} on Page~\pageref{fig:webclient}).
%
Since \ei was designed mainly for applications that process programs,
e.g., program analysis tools, this area is typically dedicated for a
tree-view of program entities, e.g., method names, class names, etc.
%
The idea is that, in addition to the input files, the user will select
some of these entities to indicate, for example, where the analysis
should start from or which parts of the program to analyze.
%
Next we explain how we can pass these selected entities to an
application.

By default the web-client is configured to work with C programs, and
thus if you open such a program (from the file-manager) and then click
on the \refreshoutline button, you will get a tree-view of this
program entities, e.g., method names (if you use \refreshoutline from
the context menu in the file-manager you will get a tree-view of
program entities for all files in the sub-tree).
%
Note that to generate this tree-view the web-client actually executes
a ``hidden'' application that is installed on the server, namely
\texttt{server/bin/default/coutline.sh}, but this is not relevant to
our discussion now (see Section~\ref{ch:clients:web:outline} for more
details).
%
Note also
that \texttt{coutline.sh} is limited and will not works perfectly for
any C program: it simply looks for lines that start with \texttt{int}
or \texttt{void} followed by something of the form
\texttt{name(...)}. 
%
This script is provided just to give an idea on how an application
that generates an outline is connected to the web-client (see
Section~\ref{ch:clients:web:outline} for more details).

As in the case of input files, the web-client always passes the
selected entities to the server when the \applybutton button is
clicked, and it is our responsibility to indicate how these entities
should be passed to our application. 
%
Let us assume that \texttt{myapp.sh} is prepared to receive entities
using a ``\texttt{-e}'' parameter as follows:

\medskip
\begin{lstlisting}
> myapp.sh -f file1.c file2.c file3.c -e sum.c:main sum.c:sum
\end{lstlisting}

\medskip 
\noindent
In order to tell the server to pass the entities (that were received
from the client) to our application, open \texttt{myapp.cfg} and
change the command-line template to the following:

\medskip
\begin{lstlisting}
<cmdlineapp>./bin/myapp.sh -f _ei_files -e _ei_outline</cmdlineapp>
\end{lstlisting}

\medskip
\noindent
As in the case of files, before executing the above command-line the
server will replace \lst{_ei_outline} by the list of received
entities.  Let us now change \texttt{myapp.sh} to process these
entities in some way, e.g., printing them on the standard output. Open
\texttt{myapp.sh} and add the following lines at the end:

\medskip
\begin{lstlisting}[style=script]
echo ""
echo "Selected entities:"
echo ""
for e in $entities 
do
   echo "- $e"
done
\end{lstlisting}
%$

\medskip
\noindent
This code simply print the entities in separated lines. Again, the
external script \texttt{parse\_params.sh} parses the command-line and
stores the list of entities in the variable \texttt{entities}.

Go back to the web-client, reload the page, select some files and
entities and execute the \texttt{MFA} application to see the result of
the last changes. It is always recommended to check that the
application works correctly from a shell first.

\section{Passing Parameters to Your Application}

In addition to input files and outline entities, real applications
receive other parameters to control different aspects. In this section
we explain how to declare parameters in the \ei framework such that
(1) they automatically appear in the web-client (or any other client)
so the user can set their values; and (2) the selected values are
passed to the application when executed.

Let us start by modifying \texttt{myapp.sh} to accept some
command-line parameters: we add a parameter ``\texttt{-s}'' to
indicate if the received outline entities should be printed; and
``\texttt{-c W}'' that takes a value \texttt{W} to indicate what to
count in each file --- here \texttt{W} can be ``\texttt{lines}'',
``\texttt{words}'' or ``\texttt{chars}''.
%
For example, \texttt{myapp.sh} could then be invoked as follows:


\medskip
\begin{lstlisting}
> myapp.sh -f file1.c file2.c file3.c -e sum.c:main sum.c:sum -s -c words
\end{lstlisting}

\medskip
\noindent
To support these parameters, change the content of \texttt{myapp.sh}
to the following:

\medskip
\begin{lstlisting}[style=script]
#!/bin/bash

. default/parse_params.sh

echo "I've received the following command-line parameters:"
echo ""
echo "  $@"

echo ""
echo "File statistics:"
echo ""

case $whattocount in
    lines) wcparam="-l"
    ;;
    words) wcparam="-w"
    ;;
    chars) wcparam="-m"
    ;;
esac

for f in $files 
do
    echo " - $f has " `wc $wcparam $f | awk '{print $1}'` $whattocount
done

if [ $showoutline == 1 ]; then
    echo ""
    echo "Selected entities:"
    echo ""
    for e in $entities 
    do
       echo "- $e"
    done
fi
\end{lstlisting}
%$

\medskip
\noindent
Compared to the previous script, you can notice that: we added lines
13-20 to take the value of ``\texttt{-c}'' into account when calling
\texttt{wc} at line 24; and in lines 27-35 we wrapped the loop that
prints the outline entities with a condition.
%
Note that \texttt{parse\_params.sh} sets the variable
\texttt{whattocount} to the value of ``\texttt{-c}'', and sets
\texttt{showoutline} to 1 if ``\texttt{-s}'' is provided in the
command-line and to 0 otherwise. Before proceeding to the next step,
execute the script from a shell to make sure that it works correctly.

Our goal is show these parameters in the web-client (or any other
client), so the user can select the appropriate values before
executing the application. The \ei framework provides an easy way to
do this, all we have to do is to modify \texttt{myapp.cfg} to include
a description of the supported parameters. Open \texttt{myapp.cfg} and
add the following inside the \lst{app} tag (e.g., immediately after
\lst{execinfo}):

\medskip
\begin{lstlisting}
  <parameters prefix = "-" check="false">
    <selectone name="c">
      <desc>
        <short>What to count</short>
        <long>Select the elements you want to count in each input file</long>
      </desc>
      <option value="lines">
        <desc>
          <short>Lines</short>
            <long>Count lines</long>
        </desc>
      </option>
      <option value="words">
        <desc>
          <short>Words</short>
          <long>Count words</long>
          </desc>
      </option>
      <option value="chars" >
        <desc>
          <short>Chars</short>
          <long>Count characters</long>
        </desc>
      </option>
      <default value="lines"/>
    </selectone>
    <flag name="s">
      <desc>
        <short>Show outline</short>
        <long>Show the selected outline entities</long>
      </desc>
      <default value="false"/>
    </flag>
  </parameters>
\end{lstlisting}

\medskip
\noindent
Let us explain the different elements of the above XML snippet. 
%
The tag \lst{parameters} includes the definition of all
parameters. The attribute \lst{prefix} is used to specify the symbol
to be attached to the parameter name when passed to the application,
for example, if we declare a parameter with name ``c'' the server will
pass it to the application as ``-c''. Note that this attribute can be
overridden by each parameter.
%
The attribute \lst{check} tells the server to check the correctness of
the parameters before passing them to the application, i.e., that they
are valid values, etc.
%
The tag \lst{selectone} defines a parameter with \lst{name} ``c'' that
can take one value from a set of possible ones. For example, the
web-client will view it as a ComboBox.
%
The \lst{desc} environment contains a text describing this parameter
and is used by the client when viewing this parameter graphically.
%
The \lst{option} tags define the valid values for this parameter, from
which one can be selected, and the \lst{default} tag defines the
default value.  The \lst{desc} environment of each \lst{option}
contains a text describing this option, e.g., the \lst{short}
description is used for the text in the corresponding ComboBox.
%
The tag \lst{flag} defines a parameter with name ``s''. This parameter
has no value, it is either provided in the command-line or not, and
its \lst{default} value is \lst{"false"}, i.e., not provided. For the
complete set of parameters supported in \ei see
\xmlstructref{server}{parameters} in Chapter~\ref{ch:server}.

Go to the web-client, reload the page, and click on the \settingbutton
button and look for the tab with the title \texttt{MFA}.  You will now
see the parameters declared above in a graphical way where you can set
their values as well.  When you click on the \applybutton button, the
web-client will pass these parameters to the server, however, we still
have to tell the server how to pass these parameters to
\texttt{myapp.sh}. Open \texttt{myapp.cfg} and change the
\lst{cmdlineapp} template to the following:

\medskip
\begin{lstlisting}
<cmdlineapp>./bin/myapp.sh -f _ei_files -e _ei_outline _ei_parameters</cmdlineapp>
\end{lstlisting}

\medskip
\noindent
As in the case of \lst{_ei_files} and \lst{_ei_outline}, the server
will replace \lst{_ei_parameters} by the list of received parameters before
executing the command-line. Execute the \texttt{MFA} application from
the web-client with different values for the parameters to see how the
output changes.

\section{Using the \ei Output Language in Your Application}

In the example that we have developed so far, the web-client simply
printed the output of \texttt{myapp.sh} in the console area. This is
the default behavior of the web-client if the output does not follow
the \ei Output Language (\eiol), which is a text-based is language
that allows generating more sophisticated output such as highlighting
lines, adding markers, etc.
%
In this section we will explain the basics of this language by
extending \texttt{myapp.sh} to use it, for more details see
Chapter~\ref{ch:eiol}.

An output in \eiol is an XML structure of the following form:

\medskip
\noindent
\begin{lstlisting}
<eiout> 
 <eicommands>
    $\xmlstructref{eiout}{eicommand}$*
 </eicommands>
 <eiactions>
    $\xmlstructref{eiout}{eiaction}$*
 </eiactions>
</eiout>
\end{lstlisting}

\medskip
\noindent
where (1) \lst{eiout} is the outermost tag that includes all the
output elements; (2) \xmlstructref{eiout}{eicommand}* is a list of commands
to be executed; and (3) \xmlstructref{eiout}{eiaction}* is a list of actions
to be declared.
%
An \xmlstructref{eiout}{eicommand} is an instruction like: \emph{print a text
  on the console}, \emph{highlight lines 5-10}, \emph{add marker at
  line 5}, etc.
%
An \xmlstructref{eiout}{eiaction} is an instructions like: \emph{when the
  user clicks on line 13, highlight lines 20-25}, etc.
%
In the rest of this section we discuss some commands and actions that
are supported in \eiol, for the complete list see
Chapter~\ref{ch:eiol}.

\subsection{Printing in the Console Area}

Recall that when the \eiol is used, the web-client does not redirect
the output to the console and thus we need a command to
print in the console area.
%
The following is an example of a command that prints ``Hello World''
in the console area:

\medskip
\begin{lstlisting}
<printonconsole consoleid="1" consoletitle="Welcome">
  <content format="text">
    Hello World
  </content>
</printonconsole>
\end{lstlisting}

\medskip
\noindent
The value of the \lst{consoleid} attribute is the console identifier
in which the given text should be printed (e.g., in the web-client the
console area has several tabs, so the identifier refers to one of
those tabs). If a console with such identifier does not exist yet, a
new one, with a title as specified in \lst{consoletitle}, is
created. If \lst{consoleid} is not given the output goes to the
default console.
%
Inside \lst{printonconsole} we can have several \lst{content} tags
which include the content to be printed (in the above example we have
only one). The attribute \lst{format} indicates the format of the
content. In the above example it is plain 'text', other formats are
supported as well, e.g., 'html'.
%

Let us change \texttt{myapp.sh} to print the different parts of its
output in several consoles. Open \texttt{myapp.sh} and change its
content to the following:


\medskip
\begin{lstlisting}[style=script]
#!/bin/bash

. default/parse_params.sh

echo "<eiout>"
echo "<eicommands>"
echo "<printonconsole>"
echo "<content format='text'>"
echo "I've received the following command-line parameters:"
echo ""
echo "   $@"
echo "</content>"
echo "</printonconsole>"

echo "<printonconsole consoleid='stats' consoletitle='Statistics'>"
echo "<content format='html'>"
echo "File statistics:"
echo "<div>"
echo "<ul>"

case $whattocount in
    lines) wcparam="-l"
    ;;
    words) wcparam="-w"
    ;;
    chars) wcparam="-m"
    ;;
esac

for f in $files 
do
    echo " <li> $f has " `wc $wcparam $f | awk '{print $1}'` $whattocount "</li>"
done
echo "</ul>"
echo "</div>"
echo "</content>"
echo "</printonconsole>"

if [ $showoutline == 1 ]; then
    echo "<printonconsole consoleid='outline' consoletitle='Outline'>"
    echo "<content format='html'>"
    echo ""
    echo "Selected entities:"
    echo "<ul>"
    echo ""
    for e in $entities 
    do
      echo "<li> $e </li>"
    done
    echo "</ul>"
    echo "</content>"
    echo "</printonconsole>"
fi
echo "</eicommands>"
echo "</eiout>"
\end{lstlisting}
%$

\medskip
\noindent
The output of \texttt{myapp.sh} is give in \eiol, because at Line 5 we
start the output with the tag \lst{eiout} which we close at line 55.
%
At Line 6 we start an \lst{eicommands} tag, inside \lst{eiout}, which
we close at Line 54.
%
Inside \lst{eicommands} we have 3 \lst{printonconsole} commands:
%
the first one is generated by lines 7-13; the second by lines 15-37;
and the last one by lines 40-52.
%
Note that the first command uses the default console, while the last
two use different consoles. Note also that the content in the last two
is given in HTML. 
%
Execute \texttt{myapp.sh} in a shell first to check that it works
correctly, and then execute the \texttt{MFA} application from the
web-client to see the effect of these changes.

\subsection{Adding Markers}

Next we explain a command for adding a marker next to a code line in
the editor area. The following is an example of such command:

\medskip
\begin{lstlisting}
<addmarker dest="path" outclass="info">
  <lines>
    <line from="4" />
  </lines>
  <content format='text'>
    (*text to associated to the marker*)
  </content>
</addmarker>
\end{lstlisting}

\medskip
\noindent
The attribute \lst{dest} indicates the \emph{full path} of the file in
which the marker should be added.
%
The attribute \lst{outclass} indicates the nature of the marker, which
can be 'info', 'error', or 'warning'. This value typically affects the
type/color of the icon to be used for the marker.
%
The tag \lst{lines} includes the lines in which markers should be
added, each line is give using the tag \lst{line} where the \lst{from}
attribute is the line number~(\lst{line} can be used to define a
region in other commands, this is why the attribute is called
\lst{from}).
%
The text inside the \lst{content} tag is associated to the marker (as
a tooltip, a dialog box, etc., depending on the client).

Let us modify \texttt{myapp.sh} to add a marker at Line 1 of each file
that it receives. Open \texttt{myapp.sh} and add the following code
snippet immediately before 54 of the previous script (i.e.,
immediately before closing the \lst{eicommands} tag):

\medskip
\begin{lstlisting}[style=script]
for f in $files 
do
  echo "<addmarker dest='$f' outclass='info'>"
  echo "<lines><line from='1'/></lines>"
  echo "<content format='text'> text for info marker of $f</content>"
  echo "</addmarker>"
done
\end{lstlisting}
%$

\medskip
\noindent
Lines 3-6 generate the actual command to add a marker for each file
passed to \texttt{myapp.sh}.
%
Execute \texttt{myapp.sh} in a shell first to check that it works
correctly, and then execute the \texttt{MFA} application from the
web-client to see the effect of these changes.

\subsection{Highlighting Code Lines}

The following command can be used to highlight code lines:

\medskip
\begin{lstlisting}
<highlightlines dest="path" outclass="info" > 
  <lines>
    <line from="5" to="10"/>
  </lines>
</highlightlines>
\end{lstlisting}

\medskip
\noindent
Attributes \lst{dest} and \lst{outclass} are as in the \lst{addmarker}
command. Each \lst{line} tag define a region to be highlighted. E.g.,
in the above example it highlights lines 5-10. You can also use the
attributes \lst{fromch} and \lst{toch} to indicate the columns in
which the highlight starts and ends respectively.

Let us modify \texttt{myapp.sh} to highlight lines 5-10 of each file
that it receives. Open \texttt{myapp.sh} and add the following code
snippet immediately before the instruction that closes the
\lst{eicommands} tag:

\medskip
\begin{lstlisting}[style=script]
for f in $files 
do
  echo "<highlightlines dest='$f' outclass='info'>"
  echo "<lines><line from='5' to='10'/></lines>"
  echo "</highlightlines>"
done
\end{lstlisting}
%$

\medskip
\noindent
Execute \texttt{myapp.sh} in a shell first to check that it works
correctly, and then execute the \texttt{MFA} application from the
web-client to see the effect of these changes.

\subsection{Adding Inline Markers}

Inline markers are widgets placed inside the code. They typically
include some read-only text. The following command adds an inline
marker:

\medskip
\begin{lstlisting}
<addinlinemarker  dest="path" outclass="info"> 
    <lines><line from="15" /></lines>
    <content format="text">
       (* Text to be viewed in the inline marker *)
    </content>
</addinlinemarker>
\end{lstlisting}

\medskip
\noindent
Attributes \lst{dest} and \lst{outclass} are as in the \lst{addmarker}
command. Each \lst{line} tag defines a line in which a widget, showing
the text inside the \lst{content}, is added. Note that some client,
e.g., the web-client, allow only plain 'text' content.

Let us modify \texttt{myapp.sh} to add an inline marker at line 15 of
each file that it receives. Open \texttt{myapp.sh} and add the
following code snippet immediately before the instruction that closes
the \lst{eicommands} tag:

\medskip
\begin{lstlisting}[style=script]for f in $files 
do
  echo "<addinlinemarker dest='$f' outclass='info'>"
  echo "  <lines><line from='15' /></lines>"
  echo "  <content format='text'> Awesome line of code!!</content>"
  echo "</addinlinemarker>"
done
\end{lstlisting}
%$

\medskip
\noindent
Execute \texttt{myapp.sh} in a shell first to check that it works
correctly, and then execute the \texttt{MFA} application from the
web-client to see the effect of these changes.


\subsection{Opening a Dialog Box}

The following command can be used to open a dialog box with some
content:

\medskip
\begin{lstlisting}
<dialogbox outclass="info" boxtitle="Done!" boxwidth="100" boxheight="100"> 
  <content format="html">
    (* text to be shown in the dialog box *)
  </content>
</dialogbox>
\end{lstlisting}

\medskip 
\noindent 
The dialog box will be titled as specified in \lst{boxtitle}, and it
will include the content as specified in the \lst{content}
environments. The attributes \lst{boxwidth} and \lst{boxheight} are
optional, they determine the initial size of the window.

Let us modify \texttt{myapp.sh} to open a dialog box with some
message. Open \texttt{myapp.sh} and add the following code snippet
immediately before the instruction that closes the \lst{eicommands}
tag:

\medskip
\begin{lstlisting}[style=script]
echo "<dialogbox outclass='info' boxtitle='Done!'  boxwidth='300' boxheight='100'>"
echo "  <content format='html'>"
echo "    The <span style='color: red'>MFA</span> analysis has been applied."
echo "    You can see the output in the result in the text area and the corresponding"
echo "    program files."
echo "  </content>"
echo "</dialogbox>"
\end{lstlisting}

\medskip
\noindent
Execute \texttt{myapp.sh} in a shell first to check that it works
correctly, and then execute the \texttt{MFA} application from the
web-client to see the effect of these changes.


\subsection{Adding Code Line Actions}

A \emph{code line action} defines a list of commands to be executed when the
user clicks on a line of code (more precisely, on a marker placed next
to the line). The commands can be any of those seen above. The
following is an example of such action:

\medskip
\begin{lstlisting}
<oncodelineclick dest="/Examples_1/iterative/sum.c" outclass="info" >
  <lines><line from="17" /></lines>
  <eicommands>
    <highlightlines>
      <lines>
        <line from="17" to="19"/>
      </lines>
    </highlightlines>
    <dialogbox boxtitle="Hey!"> 
      <content format="html">
       (* Click on the marker again to close this window *)
      </content>
    </dialogbox>
  </eicommands>
</oncodelineclick>
\end{lstlisting}

\medskip
\noindent
First note that the above XML should be placed inside the
\lst{eiactions} tag (that we have ignored so far).
%
When the above action is executed, by the web-client for example, a
marker (typically an arrow) will be shown next to line 17 of the file
\texttt{/Examples\_1/iterative/sum.c}.
%
Then, if the user clicks on this marker the commands inside the
\lst{eicommands} tag will be executed, and if the user clicks again
the effect of these commands is undone.
%
In the above case a click highlights lines 17-19 and opens a dialog
box, and another click removes the highlights and closes the dialog
box.
%
Note that the commands inside \lst{eicommands} inherit the \lst{dest}
and \lst{outclass} attributes of \lst{oncodelineclick}, but one can
override them, e.g., if we add
\lst{dest="/Examples\_1/iterative/fact.c"} to the \lst{highlightlines}
command then a click highlights lines 17-19 of \texttt{fact.c} instead
of \texttt{sum.c}.
%

Let us modify \texttt{myapp.sh} to open add a code line action, as the
one above, for each file that it receives. Open \texttt{myapp.sh} and
add the following code snippet immediately before the instruction that
closes the \lst{eiout} tag (i.e., after closing \lst{eicommands}):

\medskip
\begin{lstlisting}[style=script]
echo "<eiactions>"

for f in $files 
do
  echo "<oncodelineclick dest='$f' outclass='info' >"
  echo "<lines><line from='17' /></lines>"
  echo "<eicommands>"
  echo "<highlightlines>"
  echo "<lines><line from='17' to='19'/></lines>"
  echo "</highlightlines>"
  echo "<dialogbox boxtitle='Hey!'> "
  echo "<content format='html'>"
  echo "Click on the marker again to close this window"
  echo "</content>"
  echo "</dialogbox>"
  echo "</eicommands>"
  echo "</oncodelineclick>"
done

echo "</eiactions>"
\end{lstlisting}


\medskip
\noindent
Note that at line 1 we open the tag \lst{eiactions} and at line 21 we
close it. The rest of the code simply prints a code line action as the
one above for each file.
%
Execute \texttt{myapp.sh} in a shell first to check that it works
correctly, and then execute the \texttt{MFA} application from the
web-client to see the effect of these changes.

\subsection{Adding OnClick Actions}

OnClick actions are similar to code line actions. The difference is
that instead of assigning the action to a line of code, we can assign
it to any HTML tag that we have generated.
%
For example, suppose that at some point the application has generated
the following content in the console area:

\medskip
\begin{lstlisting}
<content format="html"/>
   <span style="color: red;" id="err1">10 errors</span> were found in the file sum.c
</content>
\end{lstlisting}

\medskip
\noindent
Note that the text ``10 errors'' is wrap by a span tag with an
identifier \texttt{err1}. The OnClick action can assign a list of
commands to be executed when this text is clicked as follows:

\begin{lstlisting}
<onclick>
   <elements>
     <selector value="#err1"/>
  </elements>
  <eicommands>
    <dialogbox boxtitle="Errors"> 
      <content format="html">
       (* There are some variables used but not declated *)
      </content>
    </dialogbox>
  </eicommands>
</onclick>
\end{lstlisting}

\medskip
\noindent
It is east to see that this action is very similar to
\lst{oncodelineclick}, the difference is that instead of \lst{lines} we
now use \lst{elements} to identify those HTML elements a click on
which should execute the commands. 

Let us modify \texttt{myapp.sh} to open add an OnClick action assigned
to the list of files that it prints on the console. First look for the
first occurrence of 

\medskip
\begin{lstlisting}[style=script]
echo "<ul>"
\end{lstlisting}

\medskip
\noindent
which should be at line 19, and change it by

\medskip
\begin{lstlisting}[style=script]
echo "<ul style='background: yellow;' id='files'>"
\end{lstlisting}

\medskip
\noindent
This change will give the list of files that we print in the console
the identifier \texttt{files}, and will change its background color to
yellow. Next add the following code immediately before the instruction
that closes \lst{eiactions}:

\medskip
\begin{lstlisting}[style=script]
echo "<onclick>"
echo "<elements>"
echo "<selector value='#files'/>"
echo "</elements>"
echo "<eicommands>"
echo "<dialogbox boxtitle='Errors'> "
echo "<content format='html'>"
echo "There are some variables used but not declated"
echo "</content>"
echo "</dialogbox>"
echo "</eicommands>"
echo "</onclick>"
\end{lstlisting}

\medskip
\noindent
Execute \texttt{myapp.sh} in a shell first to check that it works
correctly, and then execute the \texttt{MFA} application from the
web-client to see the effect of these changes. In particular, clicking
on the list of files in the console area (anywhere in the yellow
region) should open a dialog box.

