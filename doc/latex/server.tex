{% -*- mode: LaTeX; TeX-PDF-mode: t; TeX-master: "manual"; -*-
}
\chapter{Easy Interface Server}

This chapter describes the server side of the \ei platform. In
particular how to install it on an HTTP server and set some
parameters, how to make applications (\apps) available, and how to
make examples available.


\section{Installing the Server}
 
All you need to make the directory \lstinline{server} accessible
through the HTTP server, and moreover PHP access must be guaranteed
to this directory.

\fxwarning{Revise this section, and explain the installation script once it
  is ready}


\section{Configuring the Applications Server}
\label{server:sec:config}

The applications server requires a configuration file that defines
some setting parameters, available applications, and available
examples.
%
By default, it uses \filename{config/eiserver.cfg}, however, this can
be changed by editing the \filename{server/EIConfig.php}.
%
The content of this file should adhere to the
$\xmlstructrefwb{eiserver}$ XML structure as described in
Section~\ref{sec:server:xmfref}

\section{Configuration XML Reference}
\label{sec:server:xmfref}

%% EISERVER
\bigskip
\xmlstruct
{eiserver}
{
%
  This XML tag should be the root of the configurations file (see
  Section~\ref{server:sec:config}).
%
  The \xmlstructref{settings} section is used for setting some global
  parameters; the \xmlstructref{apps} section defines which \apps are
  available on the server; and \xmlstructref{examples} defines which
  example groups are available on the server.
%
}
{
%
  The server configuration file (see Section~\ref{server:sec:config})
%
}


%% SETTINGS
\bigskip
\xmlstruct
{settings}
{
%
  To be done, so far we do not have any global parameter
%
}
{\xmlstructrefwb{eiserver}}

%% EXAMPLES
\bigskip
\xmlstruct
{examples}
{
%
 Example groups available on the server
%
}
{\xmlstructrefwb{eiserver}}

%% EXAMPLES
\bigskip
\xmlstruct
{exset}
{
%
  The \xmlstructattr{id}
  \\
  Contains diferents types of elements
%
}
{\xmlstructrefwb{examples}}

%% EXAMPLE element
\bigskip
\xmlstruct
{exelements}
{
%
  Diferent types of element
%
}
{\xmlstructrefwb{exset}, \xmlstructrefwb{folder}}
%% Folder
\bigskip
\xmlstruct
{folder}
{
%
  Default values:\\
  name -> internal

%
}
{\xmlstructrefwb{exelements}}

%% File
\bigskip
\xmlstruct
{file}
{
%
  Default values:\\
  name -> internal

%
}
{\xmlstructrefwb{exelements}}

%% GitHub
\bigskip
\xmlstruct
{github}
{
%
  Default values:\\
  branch -> ``master''\\
  path -> ``/ '' \\
  internal -> ``owner + @ + repo''\\
  name -> internal
  
%
}
{\xmlstructrefwb{exelements}}

%% APPS
\bigskip
\xmlstruct
{apps}
{
%
  Defines a list of \apps (to be available on the \ei server). Each
  \app is defined by an $\xmlstructrefwb{APP}$ XML structure.
%
}
{\xmlstructrefwb{eiserver}}

%% APP
\bigskip
\xmlstruct
{app}
{
%
  Defines an app, where the different parts are as follows:
%
  \begin{itemize}

  \item \xmlstructattr{id} is an identifier used when referring to the
    \app, it should be unique among the \apps available on the same
    server.

  \item \xmlstructattr{visible} indicates if the \app should be listed
    when the list of available \apps is requested, by default it is
    \xmlstructvalue{true}. Note that even if an app is not
    \emph{publicly} visible, it can be used like any other \app (you
    just need to know its identifier).

  \item \xmlstructref{appinfo} provides some general information about
    the \app, like title, logo, description, etc.

  \item \xmlstructref{apphelp} is a (formatted) text that provides
    enough information on how the \app can be used, etc. It will be
    mainly used in the help sections of the different clients.

  \item \xmlstructref{execinfo} defines how the tool can be executed
    (e.g., command-line, server). This information is used by the
    \apps server to forward execution requests to corresponding \apps.

  \item \xmlstructref{parameters} defines the parameters accepted by
    the \app. This parameters are usually passed to client, who will
    allow users to select the appropriate values, and when and
    execution request is sent to the \apps server, these options are
    passed to the corresponding \app.

  \item \xmlstructref{permission} defines some usage permissions for
    the \app, e.g., who have permission to execute this app. By
    default any one can execute.

  \end{itemize}
%
}
{\xmlstructrefwb{apps}}


%% APPINFO
\bigskip
\xmlstruct
{appinfo}
{
%
  Provides general information on an app:

  \begin{itemize}
  \item \xmlstructref{acronym} is an acronym for the app,
    e.g., COSTA;
  \item \xmlstructref{title} is a title for the app ;
  \item \xmlstructref{logo} is an image to be used as the
    app's logo; and
  \item \xmlstructref{desc} is a (general) description of the tool.
  \end{itemize}
%
}
{\xmlstructrefwb{app}}


%% ACRONYM
\bigskip
\xmlstruct
{acronym}
{
%
  Plain text to be used as an acronym, e.g., COSTA.
%
}
{\xmlstructrefwb{appinfo}}


%% TITLE
\bigskip
\xmlstruct
{title}
{
%
  Plain text deciding a title. Typically more informative than an acronym (see 
  $\xmlstructref{acronym}$.
%
}
{\xmlstructrefwb{appinfo}}


%% LOGO
\bigskip
\xmlstruct
{logo}
{
%
  A link to an image --- in some standard format such
  \xmlstructvalue{png}, \xmlstructvalue{jpg} or \xmlstructvalue{gif}
  --- to be used by clients as a logo (for an \app). The size of the
  image typically has equal height and width (but this is not a
  requirement).
%
}
{\xmlstructrefwb{appinfo}}


%% DESC
\bigskip
\xmlstruct
{desc}
{
%
  This is a description of some entities, e.g., of an \app, a
  parameter, a parameter option, etc. It consists of two parts, the
  first one is a short description, and the second is a detailed
  description. In both cases it should be plain text. Clients will
  select one of them depending on the intended use. The description is
  typically high-level.
%
}
{\xmlstructrefwb{appinfo}, \xmlstructrefwb{selectone}, \xmlstructrefwb{selectmany}}


%% APPHELP
\bigskip
\xmlstruct
{apphelp}
{
%
  A (formatted) text that provides enough information on how an \app
  can be used, etc. It can be provided in several formats, e.g., html
  or plain text, by using several \xmlstructref{content} tags. Client
  are supposed to pick the appropriate format. It is recommended to
  always include a content in plain text since it can be view in any
  client.
%
}
{\xmlstructrefwb{app}}


%% CONTENT
\bigskip
\xmlstruct
{content}
{
%
  A text given in a specific \xmlstructattr{format}, e.g.,
  \xmlstructvalue{"text"}, \xmlstructvalue{"html"}, etc. If the
  attribute \xmlstructattr{format} is not provided, then it is assumed
  to be \xmlstructvalue{"text"} format (plain text).
%
}
{\xmlstructrefwb{apphelp}, \xmlstructrefwb{eicommand}}

%% EXECINFO
\bigskip
\xmlstruct{execinfo}
{
%
  Provides information on how to execute an app. It can be
  either a command-line app \xmlstructref{cmdlineapp}, or a
  server based app \xmlstructref{serverapp}. 
%
}
{\xmlstructrefwb{app}}

%% CMDLINEAPP
\noindent
\xmlstruct{cmdlineapp}
{
%
  Describes how to run a command-line \app, where
  \xmlstructdef{command\_template} is a template describing a
  command-line. It is best explained by an example. Consider the
  following template example

  \bigskip
  ~~~
  \fbox{\small\texttt{/path-to/app \xmlstructtemplate{_ei_files} -m \xmlstructtemplate{_ei_outline}  \xmlstructtemplate{_ei_parametes}}}
  
  \bigskip
  \noindent
  In this template, anything that starts with \xmlstructtemplate{_ei}
  is a template parameter that will be replaced by some corresponding
  information, and \texttt{/path-to/app} is the \app's executable.
  % 
  When the server receives a request for executing the corresponding
  \app, the request includes several data that should passed to the
  \app. For example, the following are typical data that should be
  passed to an \app is:
  %
  \begin{enumerate}
  \item files to be processed (e.g., program to be analysed);
  \item entities selected from the program outline (e.g., methods); and
  \item values for the different parameters.
  \end{enumerate}
  %
  The server passes this data to the \app by replacing the template
  parameters with corresponding data as follows:
  % 
  \begin{enumerate}
  \item the files are stored locally (e.g., in \texttt{/tmp}), and
    \xmlstructtemplate{_ei_files} is replaced by a list file names
    (each with an absolute path, separated by a space);
  \item \xmlstructtemplate{_ei_outline} is replaced by a list of
    selected entities (e.g., method names) --- see TODO; and
  \item \xmlstructtemplate{_ei_parametes} is replaced by the list of
    parameters generated from those provided in the request (see TODO).
  \end{enumerate}
  % 
  This result in, for example, the following instance of the template:

  \bigskip
  \hspace{0.7cm}
  \fbox{\texttt{/path-to/app \xmlstructtemplate{a.java b.java} -m \xmlstructtemplate{a.main} \xmlstructtemplate{-v 1 -d 3 -a}}}

  \bigskip 
  \noindent
  which is then executed and its output is redirected to the
  client. The server does some security checks to guarantee that the
  command-line is not executed in a bad way (see Section~\ref{})

  \bigskip 
  \noindent
  The following is a list of template parameters:

%%
  \begin{itemize}
  \item \xmlstructtemplate{_ei_files} which is replaced by a list of
    file names (separated by space) in the local file system;
\item \xmlstructtemplate{_ei_dirs} which is replaced by a list of
  dirs in the local file system;
\item \xmlstructtemplate{_ei_root} which is replaced by the local
  temporary directory name, where all files have been saved;
%%
  \item \xmlstructtemplate{_ei_outline} which is replaced by a list of
    selected entities (separated by space);
%%
  \item \xmlstructtemplate{_ei_parametes} which is replace by a
    corresponding list of parameters --- see Section~\ref{} for
    information on how this list is constructed;
%%
  \item \xmlstructtemplate{_ei_sessionid} which is replaced by a
    session identifier, this makes it possible to track the information
    of a user along several request (see Section~\ref{})
%%
  \item \xmlstructtemplate{_ei_clientid} which is replaced by a the
    client identifier, i.e., web-interface, eclipse, etc, which makes
    it possible to provide output depending on the client (not
    recommended since it double the amount of work). See
    Section~\ref{} for a list of clients.
  \end{itemize}
%%
%%   % 
}
{\xmlstructrefwb{execinfo}}

%% SERTVERAPP
\bigskip
\xmlstruct
{serverapp}
{
%
  Defines a server based application that is available at the address
  specified by the attribute \xmlstructattr{url}. In this kind of apps
  the \apps server passes the request, as it is, to the app --- See
  Section~\ref{} for more information.
%
}
{\xmlstructrefwb{execinfo}}

%% PARAMETERS
\bigskip 
\xmlstruct
{parameters} 
{
%
  Defines a list of parameters that are accepted by a corresponding
  tool. In the \xmlstructattr{prefix} attribute you can specify a
  string that will be attached to each parameter name when passed to
  the app (only in command-line mode, see
  \xmlstructrefwb{cmdlineapp}).
  % 
  For example, if \xmlstructattr{prefix}=\xmlstructvalue{"{-}{-}"} and
  we have a parameter called 'level' with value X, then '{-}{-}level
  X' will be passed to the app. The default value of
  \xmlstructattr{prefix} is \xmlstructvalue{"-"}. You can set
  \xmlstructattr{prefix} to the empty string \xmlstructvalue{""} if
  there is no need to attach anything to the option name, in such case
  the parameter name will probably include this prefix as part of its
  name.Also, you can indicate in the \xmlstructattr{check} attribute
  if you want or not that the parameters be check with your
  specification. The default value of \xmlstructattr{check} is
  \xmlstructvalue{"true"}. \xmlstructattr{check} value is like the
  \xmlstructattr{prefix} value, it is inherit on all parameters, but
  you can change on each one this properties.
%
}
{\xmlstructrefwb{appinfo}}

%% PARAM
\bigskip
\xmlstruct{param}
{
%
  Defines a parameter accepted by a corresponding app. There are
  several types of paramaters supported
%
}
{\xmlstructrefwb{parameters}}

%% SELECTONE
\bigskip
\xmlstruct
{selectone}
{
%
  Defines a parameter that takes a \emph{single} value out of a given
  list:

  \begin{itemize}
  \item \xmlstructattr{name} is the name of the parameter, it must be
    unique among all parameters of an app;

  \item \xmlstructref{desc} describes the meaning of the parameter;

  \item \xmlstructref{paramvalue}+ is a list of possible values for
    this parameter;

  \item \xmlstructref{defaultvalue} specifies the default
    value. If not specified then the first \xmlstructref{paramvalue}
    is considered as the default one;

  \item \xmlstructattr{widgetid} specifies the preferred layout when
    used in a client with a graphical interface (e.g., combo-box,
    radio button, etc.). See \xmlstructref{widgetid} for a list of
    widgets for this kind of parameter.
\item \xmlstructattr{prefix} and \xmlstructattr{check} can override the default value of these attributes, specified in \xmlstructref{parameters}, for this particular parameter.
  \end{itemize}
  %
 } {\xmlstructrefwb{param}}

%% SELECTMANY
\bigskip
\xmlstruct{selectmany}
{
%
  Defines a parameter that takes \emph{several} values out of a given
  list:

  \begin{itemize}
  \item \xmlstructattr{name} is the name of the parameter, it must be
    unique among all parameters of an app;

  \item \xmlstructref{desc} describes the meaning of the parameter;

  \item \xmlstructref{paramvalue}+ is a list of possible values for
    this parameter;

  \item \xmlstructref{defaultvalue} specifies the default
    value. If not specified then the first \xmlstructref{paramvalue}
    is considered as the default one;

  \item \xmlstructattr{widgetid} specifies the preferred layout when
    used in a client with a graphical interface (e.g., combo-box,
    radio button, etc.). See \xmlstructref{widgetid} for a list of
    widgets for this kind of parameter.
\item \xmlstructattr{prefix} and \xmlstructattr{check} can override the default value of these attributes, specified in \xmlstructref{parameters}, for this particular parameter.
  \end{itemize}
  % 
 } {\xmlstructrefwb{param}}

%% FLAG
\bigskip
\xmlstruct{flag}
{In this case the default should specify just \emph{true} or
  \emph{false}. This is ONLY to say if one parameter appears or not.
  \begin{itemize}
  \item \xmlstructattr{name} is the name of the parameter, it must be
    unique among all parameters of an app;

  \item \xmlstructref{desc} describes the meaning of the parameter;

  \item \xmlstructref{defaultvalue} specifies the default
    value. If not specified then the \emph{false} value is considered as the default one;

  \item \xmlstructattr{widgetid} specifies the preferred layout when
    used in a client with a graphical interface (e.g., combo-box,
    radio button, etc.). See \xmlstructref{widgetid} for a list of
    widgets for this kind of parameter.
\item \xmlstructattr{prefix} and \xmlstructattr{check} can override the default value of these attributes, specified in \xmlstructref{parameters}, for this particular parameter.
  \end{itemize}
}
{\xmlstructrefwb{param}}

%% TEXTFIELD
\bigskip
\xmlstruct{textfield}
{In this case the parameter is some text,this text will be passed to the app as a string or file depending on the flag \xmlstructattr{method}

\begin{itemize}
  \item \xmlstructattr{name} is the name of the parameter, it must be
    unique among all parameters of an app;

  \item \xmlstructref{desc} describes the meaning of the parameter;

  \item \xmlstructref{defaultvalue} specifies the default
    value.

  \item \xmlstructattr{widgetid} specifies the preferred layout when
    used in a client with a graphical interface (e.g., combo-box,
    radio button, etc.). See \xmlstructref{widgetid} for a list of
    widgets for this kind of parameter.
\item \xmlstructattr{prefix} and \xmlstructattr{check} can override the default value of these attributes, specified in \xmlstructref{parameters}, for this particular parameter.
  \end{itemize}
}{\xmlstructrefwb{param}}

%% HIDDEN
\bigskip
\xmlstruct{hidden}
{In this case the parameter is not showed in the app but the value is
  used like an other parameter.

\begin{itemize}
  \item \xmlstructattr{name} is the name of the parameter, it must be
    unique among all parameters of an app;

  \item \xmlstructattr{value} specifies the value.

\item \xmlstructattr{prefix} can override the default value of this attribute, specified in \xmlstructref{parameters}, for this particular parameter.
  \end{itemize}}
{\xmlstructrefwb{param}}

%% PERMISSION
\bigskip
\xmlstruct{permission}
{This is a list of users that are allowed or excluded to execute an
  app. \xmlstructattr{default} is the default value to all user that
  doesnt appear on this list. If an user appear in both list, the list
of excluded is the most important.}
{\xmlstructrefwb{app}}

%% ALLOWED
\bigskip
\xmlstruct{allowed}
{This is a list of users thar are allowed to use an app.}
{\xmlstructrefwb{permission}}

%% EXCLUDE
\bigskip
\xmlstruct{excluded}
{This is a list of users thar are excluded to use an app.}
{\xmlstructrefwb{permission}}

%% USER
\bigskip
\xmlstruct{user}
{It have an \xmlstructattr{id} of an user.}
{\xmlstructrefwb{allowed}, \xmlstructrefwb{excluded}}

\noindent
\xmlstructdef{option}


\begin{lstlisting}
<option value=$\xmlstructref{paramvalue}$>$\xmlstructref{desc}$</option>
\end{lstlisting}

\noindent
\xmlstructdef{defaultvalue}

\begin{lstlisting}
<default value=$\xmlstructref{paramvalue}$ />
\end{lstlisting}
or
\begin{lstlisting}
<default>$\xmlstructref{paramvalue}$ </default>
\end{lstlisting}

\noindent
\xmlstructdef{textformat}

( \lst{"text"} | \lst{"html"} | \lst{"svg"})

\noindent
\xmlstructdef{paramvalue}

To Be Define

\noindent
\xmlstructdef{paramname}

[a-z,A-Z,0-9,-,\_]+

\noindent
\xmlstructdef{int}

An integer value

\noindent
\xmlstructdef{bool}

( \lst{"true"} | \lst{"false"} )

\noindent
\xmlstructdef{appid}

[a-z,A-Z,0-9,-,\_]+

\noindent
\xmlstructdef{userid}

[a-z,A-Z,0-9,-,\_]+

\noindent
\xmlstructdef{widgetid}

[a-z,A-Z,0-9,-,\_]+

\noindent
\xmlstructdef{url}

A url

\noindent
\xmlstructdef{paramprefix}

Can be any string [a-z,A-Z,0-9,-,\_]+, usually it is \lst{"-"} or \lst{"--"}

\noindent
\xmlstructdef{text}

Any text

\bigskip
\bigskip
\section{Configuration JSON}

At the moment we write an example with
 all the things that you can write.

\begin{lstlisting}
 var jsonParams = {
      "command": "execute",//command
      "app_id": "costa",//app id
      "parameters": {
          "lsa":["true"],//flag parameter
          "flag2":["false"], //flag parameter
          "sone":["sla"], //selectone parameter
          "smany":["slaa","tru"], //selectmany parameter
          "_ei_files": [
              {
                  "name": "prueba",
                  "type": "directory",
                  "_ei_files" : [
                      {
                          "name":"pp.txt",
                          "content": "I'm pp.txt file \\n"
                      }//a file in a directory
                  ]//array of files and directories
              },// one directory
              {
                  "name":"f1.txt",
                  "type": "file",
                  "content": "I'm f1.txt file\\n"
              },// a file
              {
                  "name":"f2.txt",
                  "content": "I'm f2.txt file\\n"
              }// a file
          ]//array of files and directories
      }//end parameters
  };// end jsonParams

   $\$$.post("eiserver.php",
    {
       eirequest: JSON.stringify(jsonParams)
    },
    function(data) { });
\end{lstlisting}

\chapter{Outline App}

\section{Input}
I don't know the Input

\section{Output}
The Outline App returns an XML with the outline tree.

%% 
\bigskip
\xmlstruct
{outline}
{
%
  Defines a list of points to be analysed.
\begin{itemize}
  \item \xmlstructattr{text} is the name to be show.
   \item \xmlstructattr{value} is the internalname .
\item \xmlstructattr{selectable} indicates if this node can be
  selected.
\item \xmlstructattr{icon} indicates an alternative path for the icon
  of the node.

\end{itemize}
%
}
{\xmlstructrefwb{outline}}


